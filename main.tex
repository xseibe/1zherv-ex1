%%%%%%%%%%%%%%%%%%%%%%%%%%%%%%%%%%%%%%%%%%%%%%%%%%%%%%%%%%%%%%%%%%%%%%%%%%%%%%%%
% Author : [Name] [Surname], Tomas Polasek (template)
% Description : First exercise in the Introduction to Game Development course.
%   It deals with an analysis of a selected title from the point of its genre, 
%   style, and mechanics.
%%%%%%%%%%%%%%%%%%%%%%%%%%%%%%%%%%%%%%%%%%%%%%%%%%%%%%%%%%%%%%%%%%%%%%%%%%%%%%%%

\documentclass[a4paper,10pt,english]{article}

\usepackage[left=2.50cm,right=2.50cm,top=1.50cm,bottom=2.50cm]{geometry}
\usepackage[utf8]{inputenc}
\usepackage{hyperref}
\hypersetup{colorlinks=true, urlcolor=blue}

\newcommand{\ph}[1]{\textit{[#1]}}

\title{%
Analysis of Mechanics%
}
\author{%
[Name] [Surname] ([Login])%
}
\date{}

\begin{document}

\maketitle
\thispagestyle{empty}

{%
\large

\begin{itemize}

\item[] \textbf{Title:} \ph{Name of the (serious) game}

\item[] \textbf{Released:} \ph{Year of the initial release}

\item[] \textbf{Author:} \ph{Author, game studio / publisher, or N/A if unknown}

\item[] \textbf{Primary Genre:} \ph{Genre(s) of the core mechanics}

\item[] \textbf{Secondary Genre:} \ph{Additional genre(s)}

\item[] \textbf{Style:} \ph{Style of the game -- realistic, cartoon, abstract, ...}

\end{itemize}

}

\section*{\centering Analysis}

\ph{Replace all text in this section with the analysis...}

\subsection*{Instructions}

In this assignment, you are tasked with the analysis of a selected game-related title. The title may be a game, video game, serious game, or even serious application using game development tools. Your goal is to analyze the title from the point of its genres and style. As a part of this template, there are some placeholders and hints \ph{like this one}, which you should read and potentially replace with your own text.

\subsection*{Content}

After selecting the \ph{title}, you should first find out when it was \ph{first released} and who \ph{created it}. Be sure to consider the actual information if you choose a re-iteration or ``enhanced edition.'' 

Next, look at the game (or, even better, play it!) and determine the \ph{primary genre}. This genre should be the one supporting the core gameplay. You can use any genre taxonomy (not just the one from the lectures), but keep it unambiguous. A Game can have multiple modes of play -- e.g., Minecraft with creative and survival modes -- in which case you can choose any number of them, but be sure to emphasize your choice in the analysis.

After these steps, look at the \ph{secondary genres} and select one or more of them. Using Survival Minecraft as an example, we have a role-playing sandbox (primary) combined with the casual building and a hint of roguelike with the hardcore mode (secondary). Finally, determine the game's \ph{style} -- a combination of visual, aural, tactile, etc. For example, Minecraft can be considered a retro or cartoon-styled game.

Finally, move to the \ph{free-form text} part of the analysis in the form of short prose. Images should be used sparingly and best avoided them entirely. You should primarily focus on: 
\begin{enumerate}
    \item How are the primary and secondary genres reflected in the gameplay?
    \item How do the primary and secondary genre interact? Do the secondary genres support the primary genre? Do they enhance the game, or are they detrimental?
    \item Does the style support the gameplay? Why was it chosen?
\end{enumerate}

\subsection*{Formatting \& Submission}

Your submission must follow a similar \textbf{structure} as this template. You can either use the provided \LaTeX\ template or roughly replicate it in some other text processing software. The format of the analysis section is left up to you -- you can include sub-sections or write one long text. However, your whole document \textbf{must fit} on exactly one page of \textbf{A4}. The only accepted document format is \textbf{pdf}. Finally, you can submit the pdf by following the submission guidelines detailed on the \href{http://cphoto.fit.vutbr.cz/ludo/courses/izhv/exercises/sub/}{course's website}.

\end{document}
